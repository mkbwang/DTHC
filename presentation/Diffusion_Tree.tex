\documentclass{beamer}
% \setbeamertemplate{navigation symbols}{}
%\setbeamercolor{frametitle}{fg=black,bg=white}
%\setbeamercolor{title}{fg=black,bg=yellow!85!orange}
\usetheme{Boadilla}
\usepackage{makecell}
\usepackage{booktabs}
\usepackage{caption}
\usepackage{amsmath}
\usepackage{epsfig}
\usepackage{epstopdf}
\usepackage{graphicx}
\usepackage{subcaption}
\usepackage{physics}
\usepackage{hyperref}
\hypersetup{
	colorlinks=true,
	linkcolor=blue,
	filecolor=magenta,      
	urlcolor=blue,
}
\usepackage{xcolor}
\usepackage{multirow}
\usepackage{bm}
\usepackage{ragged2e}
\apptocmd{\frame}{}{\justifying}{}

\title[Bayesian Hierarchical Clustering]{Diffusion Tree Hierarchical Clustering}
\author{Mukai Wang}

\begin{document}
	\begin{frame}
		\titlepage
	\end{frame}
	
	\begin{frame}{Example Data}
		\small
		I prepare 160 data points belonging to 4 clusters. Each cluster has 40 points. The centers of the four clusters are $(2,2)$, $(2,-2)$, $(-2, 2)$ and $(-2, -2)$ accordingly. I choose two difference variance parameters to create an easy and a difficult situation.
		\begin{figure}[htbp]
			\centering
			\includegraphics[scale=0.65]{testcase.pdf}
		\end{figure}
	\end{frame}

	\begin{frame}{Performance}
		\small 
		
		For the easier case, the traditional agglomerative clustering achieved perfect performance. About 20\% of the sample trees can also achieve perfect performance. As to the more difficult case, the traditional agglomerative clustering achieved 82.5\% performance. 7 out of 100 samples achieved better performance.
		
		\begin{figure}[htbp]
			\centering
			\includegraphics[scale=0.55]{testcase_performance.pdf}
		\end{figure}
	\end{frame}
	
	\begin{frame}{Performance}
		Some bad examples stem from the fact that some individual points break away from others too early.
		\begin{figure}[htbp]
			\centering
			\includegraphics[scale=0.7]{problem1.pdf}
		\end{figure}
	\end{frame}

	\begin{frame}{Performance}
		\begin{figure}[htbp]
			\centering
			\includegraphics[scale=0.7]{problem2.pdf}
		\end{figure}
	\end{frame}

	\begin{frame}{Next Step}
		\begin{itemize}
			\item Summarize the phenomena observed, improve how to cut tree? Find MAP.
			\item Apply algorithm on CyTOF data.
		\end{itemize}
	\end{frame}
\end{document}